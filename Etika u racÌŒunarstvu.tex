  % !TEX encoding = UTF-8 Unicode
\documentclass[a4paper]{article}

\usepackage{color}
\usepackage{url}
\usepackage[T2A]{fontenc} % enable Cyrillic fonts
\usepackage[utf8]{inputenc} % make weird characters work
\usepackage{graphicx}

\usepackage[english,serbian]{babel}
%\usepackage[english,serbianc]{babel} %ukljuciti babel sa ovim opcijama, umesto gornjim, ukoliko se koristi cirilica

\usepackage[unicode]{hyperref}
\hypersetup{colorlinks,citecolor=green,filecolor=green,linkcolor=blue,urlcolor=blue}

\usepackage{listings}

%\newtheorem{primer}{Пример}[section] %ćirilični primer
\newtheorem{primer}{Primer}[section]

\definecolor{mygreen}{rgb}{0,0.6,0}
\definecolor{mygray}{rgb}{0.5,0.5,0.5}
\definecolor{mymauve}{rgb}{0.58,0,0.82}

\lstset{ 
  backgroundcolor=\color{white},   % choose the background color; you must add \usepackage{color} or \usepackage{xcolor}; should come as last argument
  basicstyle=\scriptsize\ttfamily,        % the size of the fonts that are used for the code
  breakatwhitespace=false,         % sets if automatic breaks should only happen at whitespace
  breaklines=true,                 % sets automatic line breaking
  captionpos=b,                    % sets the caption-position to bottom
  commentstyle=\color{mygreen},    % comment style
  deletekeywords={...},            % if you want to delete keywords from the given language
  escapeinside={\%*}{*)},          % if you want to add LaTeX within your code
  extendedchars=true,              % lets you use non-ASCII characters; for 8-bits encodings only, does not work with UTF-8
  firstnumber=1000,                % start line enumeration with line 1000
  frame=single,	                   % adds a frame around the code
  keepspaces=true,                 % keeps spaces in text, useful for keeping indentation of code (possibly needs columns=flexible)
  keywordstyle=\color{blue},       % keyword style
  language=Python,                 % the language of the code
  morekeywords={*,...},            % if you want to add more keywords to the set
  numbers=left,                    % where to put the line-numbers; possible values are (none, left, right)
  numbersep=5pt,                   % how far the line-numbers are from the code
  numberstyle=\tiny\color{mygray}, % the style that is used for the line-numbers
  rulecolor=\color{black},         % if not set, the frame-color may be changed on line-breaks within not-black text (e.g. comments (green here))
  showspaces=false,                % show spaces everywhere adding particular underscores; it overrides 'showstringspaces'
  showstringspaces=false,          % underline spaces within strings only
  showtabs=false,                  % show tabs within strings adding particular underscores
  stepnumber=2,                    % the step between two line-numbers. If it's 1, each line will be numbered
  stringstyle=\color{mymauve},     % string literal style
  tabsize=2,	                   % sets default tabsize to 2 spaces
  title=\lstname                   % show the filename of files included with \lstinputlisting; also try caption instead of title
}

\begin{document}

\title{Etika u računarstvu\\ \small{Seminarski rad u okviru kursa\\Metodologija stručnog i naučnog rada\\ Matematički fakultet}}

\author{
	Marina Borozan, Matija Miličević,\\
	Stefan Mirić, Nikola Vuković\\
	marinaborozanv@gmail.com, matijanme@gmail.com,\\
	stefangiggs96.sm@gmail.com, sterlu.sd@gmail.com
}

%\date{9.~april 2015.}

\maketitle

\abstract{
  Sažetak rada, kao reklama za rad. Jedan pasus, piše se na kraju. 
}

\tableofcontents

\newpage

\section{Uvod}

Uvod u temu, definisanje pojmova, cilj rada, najava ostatka rada.
Pisati pred kraj, pre apstrakta. Treba da sadrži reference.

\section{Teorije etike}
Svaka teorija etike sadrži dva obavezna elementa. Prvi je teorija koja definiše šta se smatra za dobro, odnosno vredno. Za definisanje pojma ``dobro'' teorija etike može uzeti bilo koji pojam za koji se zalaže (na primer lična sloboda ili život svih ljudi). Ipak često nije dovoljno definisati samo ovaj pojam da bi se izbegle dileme u izboru pravilnog postupka. Zato se uvek definiše i teorija ispravnog koja nalaže kako pojedinci treba da reaguju na vrednosti. 

Osvrnućemo se na neke škole etike i njihove dobre i loše strane. Pregled koji sledi nikako ne predstavlja pregled svih grana etike koji bi bio preopširan, već služi da čitaocu da ideju o različitim pristupima definisanja dobrog i ispravnog. 

\subsection{Deontologija - Kantijanizam}
(Kantianism)
%Nemački filozof Imanuel Kant (1724-1804) je za vreme svog života radio na etičkoj teoriji koja se zasnivala na pretpostavci postojanja univerzalnih moralnih zakona. Po Kantu

\subsection{Konsekvencijalizam - Utilitarizam postupaka}
(Act utilitarism)

\subsection{Konsekvencijalizam - Utilitarizam pravila}
(Rule utilitarism)

\subsection{social contract theory}
(social contract theory)

\subsection{Etika vrline}
(virtue ethics)


\section{Odnos etike i tehnološkog razvoja}

Etika je kao naučna disciplina postojala pre mnogih tehnoloških napredaka civilizacije. Kako su mogućnosti pojedinaca ili grupa rasle tako je dolazilo i do novih etičkih pitanja. Naučni razvoji u fizici i hemiji su bez sumnje doveli do poboljšanja uslova života mnogima, ali moramo biti svesni mogućnosti novih tehnologija. Primer: Da li je etički koristiti biološko ili nuklearno oružje u ratu?.
%Sa svakom novom istorijskom epohom pojavljivala su se nova etička pitanja.

Pojavom ličnih računara i interneta mnogima je pružena mogućnost da komuniciraju na velikim daljinama, vrše monetarne transakcije, dele elektronski sadržaj... Međutim, time je takođe olakšano lažno predstavljanje, prevara, deljenje neovlašćenog sadržaja i slično.

\subsection{Moderna upotreba računara}
(Za sad samo ideja)

\subsection{Internet i mrežna komunikacija}
(Takođe samo ideja)


\section{Zaključak}
\label{sec:zakljucak}

Zaključak

\addcontentsline{toc}{section}{Literatura}
\appendix
\bibliography{seminarski} 
\bibliographystyle{plain}

\appendix
\section{Dodatak}
Ovde pišem dodatne stvari, ukoliko za time ima potrebe.


\end{document}
