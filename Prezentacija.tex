\documentclass[pdf]{beamer}
\mode<presentation>{}

\usepackage[utf8]{inputenc}
\usepackage[english,serbian]{babel}

\usetheme{Madrid}
\useoutertheme{miniframes} % Alternatively: miniframes, infolines, split
\useinnertheme{circles}

%TODO Ova tema ne podržava predugačak tekst za autore + ceo podnaslov
\title{Etika u računarstvu}
\subtitle{Seminarski rad}% u okviru kursa\\Metodologija stručnog i naučnog rada\\ Matematički fakultet}
\author{Marina Borozan, Matija Miličević,\\
	Stefan Mirić, Nikola Vuković\\}

\begin{document}

\begin{frame}
	\titlepage	
	\end{frame}


\begin{frame}
\frametitle{Uvod}

\begin{itemize}
\item{Tehnoloski razvoj i nove mogucnosti}
\item{Ubrzane promene u racunarstvu}
\item{Moralne dileme i prihvatanje odgovornosti}
\item{Kratak sadrzaj:}
	\begin{itemize}
	\item{Uvod u teoriju etike}
	\item{Etika u racunarstvu}
	\item{Razvoj tehnologije i primena etike}
	\end{itemize}
\end{itemize}

\end{frame}

\begin{frame}
\frametitle{Etika}

Definicija: Oblast filozofije koja se bavi sistematizacijom, odbranom i preporukom ispravnog i pogresnog ponasanja.

Podoblasti:
\begin{itemize}
\item{Metaetika

	Bavi se temeljima etike i apstraktnim pitanjima.}
\item{Normativna etika

	Najkorisnija za temu etike u racunarstvu. Definise moralne probleme na praktican nacin.}
\item{Primenjena etika
	
	Razmatra konkretne situacije i moralna pitanja. Primenjuje principe definisane u normativnoj etici.}
\end{itemize}

\end{frame}

\begin{frame}
\frametitle{Teorije etike}

\begin{itemize}
\item{Pojam dobrog}
\item{Pojam ispravnog}
\end{itemize}

%ovde tabela podele

\end{frame}

\begin{frame}
\frametitle{Deontologija}

Naziv potice od grckih reci koje označavaju duznost (deon) i nauku (logos). Procenjuje moralnost postupaka.

%ovde slika Kanta

Kantijanizam - najpoznatija podgrana deontologije.

Principi:
\begin{itemize}
\item{Univerzalnost moralnih zakona}
\item{Dobra volja kao motiv}
\item{Pojam duznosti}
\item{Kategoricki imperativ}
\end{itemize}

\end{frame}


\begin{frame}
\frametitle{Etika i tehnološki razvoj}

	Nove tehnologije:

	\begin{itemize}

	\item Nove mogućnosti

	\item Novi problemi

	\end{itemize}

%\pause
	Stavovi o tehnološkom napretku:

	\begin{enumerate}

	\item Tehnicizam
	\item Optimizam
	\item Skepticizam

	\end{enumerate}

	\end{frame}


\begin{frame}
\frametitle{Računarska Etika}
	%Pričati o tome šta je računarska etika u sklopu priče o istoriji.
	
	Norbert Viner
	\begin{itemize}
	\item ``Kibernetika'' (1948) %Tačka iza godine?
	\item ``Ljudska upotreba ljudskih bića'' (1950)
	\end{itemize}
	
	Volter Maner
	\begin{itemize}
	\item Eksperimentalni kurs (1976)
	\item Plan kursa (1978)	
	\item Monograf (1980) %Saznaj šta je monograf
	\end{itemize}
	\end{frame}

\begin{frame}
\frametitle{Viđenja Računarske Etike}

	Pristupi u naučnoj literaturi:
	\begin{enumerate}
	\item Pristup nepostojanja (\textit{No resolution approach})
	\item Profesionalni pristup (\textit{The professional approach})
	\item Radikalni pristup (\textit{The radical approach})
	\item Konzervativni pristup (\textit{The conservative approach})
	\item Inovativni pristup (\textit{The innovative approach})
	\end{enumerate}
	\end{frame}


\begin{frame}
\frametitle{Primeri etičkih dilema u računarstvu}
		Pisanje virusa:
		\begin{itemize}
		\item Kalgari Univerzitet (Ken Barker)
			\begin{itemize}
			\item Pisanje virusa na kursu računarske sigurnosti
			\item Zabrinutost antivirusnih kompanija 
			\end{itemize}
		
		\item Sonoma Univerzitet (Džordž Ledin)
			\begin{itemize}
			\item Osnovao kurs za pravljenje bolje sigurnosne zaštite
			\item Bojkot sigurnosnih kompanija
			\end{itemize}
		\end{itemize}
\end{frame}

\begin{frame}
\frametitle{Primeri etičkih dilema u računarstvu}
	Neki dodatni primeri:
	\begin{itemize}
	\item Da li je prihvatljivo da se kupi softver pa onda instalirati ga dvaput?
	\item Šta ako ga instaliramo, a zatim napravimo 50 kopija i prodamo zainteresovanim kupcima?
	\item Da li kompanija ima pravo da čita elektronsku poštu svojih zaposlenih?
	\item Da li kompanija ima pravo da nadzire koje Web stranice njeni zaposleni posećuju?
	\item Da li korisnik ne sme da modifikuje program iako je njegov cilj da ga poboljša?
	%smisliti jos neki primer
	\end{itemize}
\end{frame}

\begin{frame}
\frametitle{Zaključak}
	(Matija)TODO
	\end{frame}


\begin{frame}
\frametitle{Literatura}
	TODO
	\end{frame}

\end{document}

