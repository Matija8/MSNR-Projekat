

 % !TEX encoding = UTF-8 Unicode

\documentclass[a4paper]{report}

\usepackage[T2A]{fontenc} % enable Cyrillic fonts
\usepackage[utf8x,utf8]{inputenc} % make weird characters work
\usepackage[serbian]{babel}
%\usepackage[english,serbianc]{babel}
\usepackage{amssymb}

\usepackage{color}
\usepackage{url}
\usepackage[unicode]{hyperref}
\hypersetup{colorlinks,citecolor=green,filecolor=green,linkcolor=blue,urlcolor=blue}

\newcommand{\odgovor}[1]{\textcolor{blue}{#1}}

\begin{document}

\title{Etika u računarstvu\\ \small{
	Marina Borozan, Matija Miličević,\\
	Stefan Mirić, Nikola Vuković\\
	marinaborozanv@gmail.com, matijanme@gmail.com,\\
	stefangiggs96.sm@gmail.com, sterlu.sd@gmail.com
}}

\maketitle

\tableofcontents

\chapter{Uputstva}
\emph{Prilikom predavanja odgovora na recenziju, obrišite ovo poglavlje.}

Neophodno je odgovoriti na sve zamerke koje su navedene u okviru recenzija. Svaki odgovor pišete u okviru okruženja \verb"\odgovor", \odgovor{kako bi vaši odgovori bili lakše uočljivi.} 
\begin{enumerate}

\item Odgovor treba da sadrži na koji način ste izmenili rad da bi adresirali problem koji je recenzent naveo. Na primer, to može biti neka dodata rečenica ili dodat pasus. Ukoliko je u pitanju kraći tekst onda ga možete navesti direktno u ovom dokumentu, ukoliko je u pitanju duži tekst, onda navedete samo na kojoj strani i gde tačno se taj novi tekst nalazi. Ukoliko je izmenjeno ime nekog poglavlja, navedite na koji način je izmenjeno, i slično, u zavisnosti od izmena koje ste napravili. 

\item Ukoliko ništa niste izmenili povodom neke zamerke, detaljno obrazložite zašto zahtev recenzenta nije uvažen.

\item Ukoliko ste napravili i neke izmene koje recenzenti nisu tražili, njih navedite u poslednjem poglavlju tj u poglavlju Dodatne izmene.
\end{enumerate}

Za svakog recenzenta dodajte ocenu od 1 do 5 koja označava koliko vam je recenzija bila korisna, odnosno koliko vam je pomogla da unapredite rad. Ocena 1 označava da vam recenzija nije bila korisna, ocena 5 označava da vam je recenzija bila veoma korisna. 

NAPOMENA: Recenzije ce biti ocenjene nezavisno od vaših ocena. Na osnovu recenzije ja znam da li je ona korisna ili ne, pa na taj način vama idu negativni poeni ukoliko kažete da je korisno nešto što nije korisno. Vašim kolegama šteti da kažete da im je recenzija korisna jer će misliti da su je dobro uradili, iako to zapravo nisu. Isto važi i na drugu stranu, tj nemojte reći da nije korisno ono što jeste korisno. Prema tome, trudite se da budete objektivni. 
\chapter{Recenzent \odgovor{--- ocena:} }


\section{O čemu rad govori?}
Upoznajemo se sa etikom: šta je etika i kako se ona deli. Neke od grana teorije etike su: deontologija, konsekvencijalizam, etika vrline. U računarstvu se javlja 40-ih godina i ona kao termin nema određenu definiciju. Postoje pristupi, principi i pravila kojih se treba pridržavati. Neke etičke dileme: Da li studenti treba da nauče da pišu viruse? Da li imamo pravo na privatnost podataka?...

\section{Krupne primedbe i sugestije}
\textbullet Sadržaj: Kada pogledamo sadržaj, u okviru 3. naslova (Teorija etike) ima mnogo više grananja na podnaslove u odnosu na ostale, to treba izmeniti.
\\\textbullet Poglavlje 2: U prvi pasus je potrebno dodati bar još jednu rečenicu (kako bi zaista to i bio), umesto toga može da se napiše “Često se podela može izvršiti:”, pa nabrojati sva tri i uz svaki staviti opis koji je dat (kao što je urađeno u poglavlju 4). 
\\\odgovor{Poglavlje je izmenjeno po ugledu na poglavlje 4.}
\\\textbullet Poglavlje 3: Nigde nije referisano u tekstu na tabelu i slike. Kada pogledamo tabelu, fali informacija na šta se koji red odnosi, treba da piše: teorija etike, vrednost, predstavnik. 
\\\odgovor{Tekst je dopunjen sa referencom na tabelu. Tabela je dopunjena sa neophodnim informacijama. }
\\\textbullet Poglavlje 3.1.1: Odakle su sve ove informacije koje se tvrde o Kantu? U drugom pasusu postoji citat, on treba da bude iskošen (eng. italic) i nema referencu. Poslednji pasus čini samo jedna rečenica.
\\\textbullet Poglavlje 3.3: Na početku i kraju poglavlja stoji po jedna rečenica. Potrebno je dodati još rečenica, kako bi to bio pasus ili preformulisati na drugačiji način.
\\\textbullet Poglavlje 6.2: Ukoliko se već vrši nabrajanje, bolje je izbaciti “zatim,” i umesto toga staviti “:” na početku, pre svega, pa tek onda da ide nabrajanje.
\\\textbullet Zaključak: Drugi pasus čini samo jedna rečenica.
\\\textbullet Literatura: U prvoj navedenoj literaturi nema naveden autor, a u većini nedostaju godine. Kod navedene knjige fali adresa. 

\section{Sitne primedbe}
\textbullet Uvod: U trećem pasusu ima višak “se”.
\\\textbullet Poglavlje 3: U 4. redu treba da piše “Ipak, ”.
\\\textbullet Poglavlje 3.1: U prvoj rečenici treba da piše “potiče”, a ne “potice”. Poslednja rečenica čudno zvuči, padeži nisu uklopljeni kako treba.
\\\textbullet Poglavlje 3.4: Drugi pasus (4. red), piše “drustvu” umesto “društvu”. U poslednjem pasusu je progutano slovo, piše “iako manje praktiča”.
\\\textbullet Poglavlje 4: U drugom pasusu (4. red), poslednji zarez je pogrešno prilepljen.
\\\textbullet Poglavlje 5: Prilikom navođenja engleskih termina fali \textit{eng.} u zagradama.
	Progutano je slovo u drugom pasusu (6. red), piše “elktronskog”.
\\\textbullet Poglavlje 5.1: Kod navođenja engleskih termina fali \textit{eng.} u zagradama.	
\\\textbullet Poglavlje 5.2: U drugom redu ima j viška, piše “određenjim”. Kod engleskih termina fali \textit{eng.} u zagradama.
\\\textbullet Poglavlje 6.1: U pretposlednjem pasusu, pre imena profesora fali “, ”, a kod navođenja engleskog termina fali \textit{eng.} u zagradi. U poslednjem pasusu piše “opšto mišljenje”, treba “opšte mišljenje”.


\odgovor{Sve navedene greške su ispravljene u radu.}


\section{Provera sadržajnosti i forme seminarskog rada}
% Oдговорите на следећа питања --- уз сваки одговор дати и образложење

\begin{enumerate}
\item Da li rad dobro odgovara na zadatu temu?\\
Rad odgovara na zadatu temu, obuhvaćeno je sve, uopštene priče o filozofiji ima i više nego što je to potrebno. Deo o etici u računarstvu je sasvim dovoljan i lepo je objašnjen.

\item Da li je nešto važno propušteno?\\
Što se tematike tiče nije propušteno ništa važno, jedino je potrebno ispraviti navedene greške.

\item Da li ima suštinskih grešaka i propusta?\\
U radu je potrebno ispraviti prethodno navedene primedbe i sugestije, ali nema previše bitnih propusta.
\item Da li je naslov rada dobro izabran?\\
 Jeste, tako glasi tema, nije bilo potrebe da se menja. Smatram da je sasvim dobar naslov, dovoljno je kratak i razumljiv je.

\item Da li sažetak sadrži prave podatke o radu?\\
 Sažetak rada sadrži sve ono što je i objašnjeno u radu, sve je obuhvaćeno.

\item Da li je rad lak-težak za čitanje?\\
 Rad je sasvim korektan za čitanje, nije previše težak, ne sadrži komplikovane izraze i detaljno je objašnjen.

\item Da li je za razumevanje teksta potrebno predznanje i u kolikoj meri?\\ 
Sam rad nam daje mogućnost da se na početku upoznamo konkretno sa filozofijom, pa nas polako uvodi u istoriju razvoja i primenu u računarstvu kroz primere. Nije potrebno previše predznanja za razumevanje, ukoliko nismo mnogo informisani možemo lako sve da razumemo, sve je dovoljno detaljno objašnjeno.

\item Da li je u radu navedena odgovarajuća literatura?\\
Navedena je potrebna literatura. Rad sadrzi naučni rad, smislenu knjigu, kao i potrebu internet stranicu. Potrebno je dodati godine pri navođenju date literature. 

\item Da li su u radu reference korektno navedene?\\
U primedbama je navedeno na kojim delovima smatram da fale reference. Nigde nema referenca na slike, kao ni referenca na tabelu, ostale su korektne.

\item Da li je struktura rada adekvatna?\\
Potrebno je lepše organizovati sadržaj, kako bi rad bio mnogo lepši i kvalitetniji. Postoje pasusi koji sadrže po jednu ili dve rečenice, njih treba drugačije organizovati ili dodati još rečenica.

\item Da li rad sadrži sve elemente propisane uslovom seminarskog rada (slike, tabele, broj strana...)?\\ Sadrži tabelu, a u okviru te tabele se nalaze 3 slike. U radu ima potreban broj strana, postoje reference na: knjigu, sajt i na naučni rad.

\item Da li su slike i tabele funkcionalne i adekvatne?\\ Tabela nam prikazuje podelu teorije etike i vrline koje su adekvatne za datu podelu. U okviru tabele imamo dodatno po sliku predstavnika za svaku od podela, pa samim tim na lak način ova podela može da se zapamti. U tabeli fali kolona sa redovima: teorija etike, vrednost, predstavnik.
\end{enumerate}

\section{Ocenite sebe}
% Napišite koliko ste upućeni u oblast koju recenzirate: 
% a) ekspert u datoj oblasti
% b) veoma upućeni u oblast
c) srednje upućeni
% d)  d) malo upućeni 
% e) skoro neupućeni
% f) potpuno neupućeni
% Obrazložite svoju odluku
  \\ Određene stvari smo učili u srednjoj školi iz filozofije, a konkretno o etici u računarstvu sam čitala u slobodno vreme.

\chapter{Recenzent \odgovor{--- ocena:} }


\section{O čemu rad govori?}

% Напишете један кратак пасус у којим ћете својим речима препричати суштину рада (и тиме показати да сте рад пажљиво прочитали и разумели). Обим од 200 до 400 карактера.
Rad govori o etici kao filozofskom pojmu i etičkom ponašanju u računarstvu. Uveden je pojam etike kao i njenih najvažnijih podgrana. Zadate su opšte smernice etičkog ponašanja u računarstvu. Prikazan je odnos etike i računarstva, i predstavljen primer u kome se postavlja pitanje da li su određeni postupci etički. Na kraju, postavljeno je par pitanja čiji odgovori mogu zastupati različita mišljenja.


\section{Krupne primedbe i sugestije}
% Напишете своја запажања и конструктивне идеје шта у раду недостаје и шта би требало да се промени-измени-дода-одузме да би рад био квалитетнији.
\begin{itemize}
    \item Pasusi pod rednim brojevima 2 i 3 previše zalaze u filozofsku definiciju etike i ni na koji način ne poboljšavaju viđenje etike u računarstvu, niti doprinose boljem razumevanju nastavka rada. Smatram da bi bilo dobro navesti primer za viđenje računarstva iz ugla svake podgrane, ili je izbaciti iz rada ukoliko se ne može primeniti na računarstvo.
    \item Tabela u radu ne sadrži adekvatne podatke niti se može nešto bitno zaključiti iz nje.
    \item Uvod i zaključak se ponavljaju. I u uvodu i u zaključku je napisano šta se obrađuje u radu.
    \item Prvi deo 5. poglavlja koji govori o kibernetici nije relevantan za zadatu temu.
\end{itemize}

\section{Sitne primedbe}
% Напишете своја запажања на тему штампарских-стилских-језичких грешки
\begin{itemize}
    \item Sažetak
    \begin{itemize}
        \item Mislim da je izražavanje u prvom licu množine neadekvatno. Možda bi bilo bolje napisati - ``\textit{Da bi se definisao i izmerio uticaj ovog snažnog pokreta, bitno je istražiti moralne implikacije novonastalih mogućnosti.}''
        
        \odgovor{TODO.}
        
    \end{itemize}
    \item Uvod
    \begin{itemize}
        \item Mislim da je neadekvatno izražavanje u prvom licu množine.
        
        \odgovor{TODO.}
        
        \item Mislim da je suvišno napisati ``svakodnevne'' i ``novootkrivene'' jedno do drugog - ``\textit{... etika nudi za \textbf{svakodnevne novootkrivene} dileme na koje možemo naići}''

        \odgovor{Preformulisano u ``za rešavanje konstantno nastajućih dilema''.}
                
        \item Suvišna reč - ``\textit{Rad će \textbf{se} u nastavku uvesti čitaoca ...}''
        
        \odgovor{Ispravljena greška, izbačeno suvišno \textbf{se}.}
        
    \end{itemize}
    \item Etika
    \begin{itemize}
        \item Zameniti ``na koje'' sa ``kojim'' - ``\textit{Metaetika se bavi proučavanjem temelja etike, načina \textbf{na koje} definišemo pojmove}''
        
        \odgovor{Ispravljeno.}
        
        \item Opis tabele navesti ispod tabele, preglednije je nego navođenje opisa iznad tabele
        
        \odgovor{Opis tabele pomeren ispod tabele.}
        
    \end{itemize}
    \item Teorija etike
    \begin{itemize}
        \item Mislim da je suvišna rečenica - ``\textit{Pored toga, teorije su često
otvorene različitim interpretacijama, pa i zavisno kako ih primenjujemo
možemo doći do različitih zaključaka.}''

		\odgovor{TODO.}
		
    \end{itemize}
    \item Deontologija
    \begin{itemize}
        \item Slovna greška - ``\textit{Termin deontologija \textbf{potice} od grčkih reči...}''
        
        	\odgovor{Greška ispravljena.}
        	
    \end{itemize}
    \item Kantijanizam
    \begin{itemize}
        \item Slovna greška - ``\textit{Prema Kantu, za \textbf{odredjeni} postupak ...}''
        \item Slovna greška - ``\textit{...postoji razumno
objašnjenje \textbf{zasto} je moralno...}''
        \item Slovna greška - ``\textit{Jedino se dobra volja
\textbf{moze} smatrati...}''

	\odgovor{Ispravljene su sve tri greške.}
	
    \end{itemize}
    \item Konsekvencijalizam
    \begin{itemize}
        \item Slovna greška - ``\textit{...različite (često međusobno
\textbf{sukobljenje}) teorije...}''

	\odgovor{Greška ispravljena.}
	
        \item Dodati reč - ``\textit{...imaju pravo da rade šta god žele kako bi maksimizovali to
zadovoljstvo [9] (u direktnom sukobu sa načelima deontologije)}'' - ispraviti u - ``\textit{...imaju pravo da rade šta god žele kako bi maksimizovali to
zadovoljstvo [9] (\textbf{što je} u direktnom sukobu sa načelima deontologije)}''

	\odgovor{Ispravljeno.}
	
    \item Greška u padežu - ``\textit{Drugim rečima,
\textbf{površno primenjivanje} utilitarizma bismo zaključili}''

	\odgovor{Greška ispravljena.}

    \end{itemize}
    \item Utilitarizam postupaka
    \begin{itemize}
        \item Greška u redosledu slova - ``\textit{...direktne posledice \textbf{ordeđenog}...}''
        
        \odgovor{Greška ispravljena.}
        
    \end{itemize}
    \item Poređenje deontologije i konsekvencijalizma
    \begin{itemize}
        \item Slovna greška u naslovu
        
        \odgovor{Greška ispravljena, \textbf{dj} prebačeno u \textbf{đ}.}
        
        \item Slovna greška - ``\textit{\textbf{Medjutim}, deontolozi se mogu \textbf{naci} u dilemi ukoliko su oba izbora
moralno ispravna, a \textbf{medjusobno} konfliktna.}''

		\odgovor{Sve tri greške su ispravljene.}

    \end{itemize}
    \item Etika vrline
    \begin{itemize}
        \item Ispraviti konstrukciju rečenice - ``\textit{...vrline ne ogledaju samo u tome da postupamo na odredjeni nacin...}''
        - ispraviti u - ``\textit{...vrline ne ogledaju u postupanju na određeni način...}'' 
        
        \odgovor{TODO.}
        
        \item Nedostaje slovo - ``\textit{Iako manje \textbf{praktiča} i određena...}''

        \odgovor{Greška ispravljena, slovo \textbf{n} je ubačeno.}

        \item Izbacio bih poslednju rečenicu jer nije relevantna.
        
        \odgovor{TODO.}
    \end{itemize}
    \item Odnos etike i tehnološkog razvoja
    \begin{itemize}
        \item Ispraviti konstrukciju rečenice - ``\textit{Međutim, time je takode olakšano lažno predstavljanje, prevara, deljenje neovlašćenog sadržaja i slično ,na načine ne
prethodno viđene.}'' - ispraviti u - 
``\textit{Međutim, time je takođe olakšano lažno predstavljanje, prevara, deljenje neovlašćenog sadržaja i slično, na potpuno nove načine.}''

\odgovor{TODO. (Šta ne valja sa originalom?)}

    \end{itemize}
    \item Etika i računari
    \begin{itemize}
        \item Slovna greška - ``\textit{Nakon završetka rata napisao je knjigu “Kibernetika”(1948) u kojoj je opisao novonastalu oblast i pomenuo neke društvene
i etičke implikacije \textbf{elktronskih} računara.}''
        \item Slovna greška - ``\textit{Njemu je delovalo da se uz računare pojavljuju etički problemi koji ne bi postojali da nije bilo izuma \textbf{elktronskog}
računara.}''

\odgovor{Obe greške su ispravljene, ubačeno slovo \textbf{e}.}

    \end{itemize}
    \item Vidjenja računarske etike
    \begin{itemize}
        \item Slovna greška u naslovu
        
        \odgovor{Greška ispravljena.}
        
    \end{itemize}
    \item Pravila etičkog ponašanja
    \begin{itemize}
        \item Slovna greška - ``\textit{Više organizacija se oprobalo u definisanju ovakvih pravila dobrog
\textbf{ponošanja}.}''
        \item Slovna greška - ``\textit{Doprinosite društvu i blagostanju ljudi, budite svesni da
smo svi članovi \textbf{računarksog}}''
        \item Slovna greška - ``\textit{Ne pretražuj \textbf{tudje} datoteke}''
        
        \odgovor{Sve tri greške su ispravljene.}
        
    \end{itemize}
    \item Primeri etičkih dilema u računarstvu
    \begin{itemize}
        \item Promeniti ``\textit{pisac}'' i ``\textit{pisači}'' virusa u ``\textit{tvorac}'' i ``\textit{tvorci}'' virusa.
        
        \odgovor{TODO.}

    \end{itemize}
    \item Dodatni primeri
    \begin{itemize}
        \item Nedostaje razmak - ``\textit{Da li je prihvatljivo da se kupi softver pa onda instalirati ga \textbf{dvaputa}}''
        
        \odgovor{Dvaputa je gramatički ispravno u ovom kontekstu. (TODO: proveri 100\%)}

        \item Nedostaje razmak - ``\textit{Šta ako ga instaliramo\textbf{,a} onda damo prijatelju da ga koristi?}''
        
        \odgovor{Greška ispravljena, razmak ubačen.}

        \item Nedostaje zarez - ``\textit{Šta ako ga instaliramo\textbf{ a} zatim napravimo 50 kopija i prodamo za-
interesovanim kupcima?}''

		\odgovor{Greška ispravljena, zarez dodat.}

    \end{itemize}
    \item Zaključak
    \begin{itemize}
        \item Pogrešno slovo - ``\textit{Pomenuli smo uticaj tehnologije na etiku i izazove koji \textbf{si} javljaju}''

        \odgovor{Greška ispravljena.}

    \end{itemize}
\end{itemize}

\section{Provera sadržajnosti i forme seminarskog rada}
% Oдговорите на следећа питања --- уз сваки одговор дати и образложење

\begin{enumerate}
\item Da li rad dobro odgovara na zadatu temu?\\
Prva polovina rada ne odgovara na zadatu temu. Druga polovina rada odgovara.
\item Da li je nešto važno propušteno?\\
Mislim da bi tema o internet prevarama mogla da doprinese boljem radu, kao i tema o nadgledanju podataka koji kruže Internetom od strane državnih obaveštajnih službi.
\item Da li ima suštinskih grešaka i propusta?\\
U radu nema suštinskih grešaka osim nekoliko slovnih i gramatičkih.
\item Da li je naslov rada dobro izabran?\\
Mislim da rad nije dobro odabran jer dobar deo rada je posvećen filozofskom pojmu etike, a ne primeni etike u računarstvu.
\item Da li sažetak sadrži prave podatke o radu?\\
Ne.
\item Da li je rad lak-težak za čitanje?\\
Rad je težak za čitanje jer je previše pažnje posvećeno filozofiji, a manje etici u računarstvu.
\item Da li je za razumevanje teksta potrebno predznanje i u kolikoj meri?\\
Nije potrebno predznanje za razumevanje teksta.
\item Da li je u radu navedena odgovarajuća literatura?\\
Jeste.
\item Da li su u radu reference korektno navedene?\\
Navođenje referenci u tekstu ne počinje od broja 1, nisu uređene reference.
\item Da li je struktura rada adekvatna?\\
Struktura rada je adekvatna ako se izuzme poglavlje 3, ono mislim da je suvišno.
\item Da li rad sadrži sve elemente propisane uslovom seminarskog rada (slike, tabele, broj strana...)?\\
Rad sadrzi sve propisane elemente.

\item Da li su slike i tabele funkcionalne i adekvatne\\
Slike se nalaze u tabeli koja ne predstavlja nikakav zaključak, opažanje, niti interpretaciju nekog rezultata. Mislim da nisu adekvatne.
\end{enumerate}
\section{Ocenite sebe}
Prvi put sam čuo za oblast etike u računarstvu i potpuno sam neupućen.
% Napišite koliko ste upućeni u oblast koju recenzirate: 
% a) ekspert u datoj oblasti
% b) veoma upućeni u oblast
% c) srednje upućeni
% d) malo upućeni 
% e) skoro neupućeni
% f) potpuno neupućeni
% Obrazložite svoju odluku


\chapter{Recenzent \odgovor{--- ocena:} }


\section{O čemu rad govori?}
% Напишете један кратак пасус у којим ћете својим речима препричати суштину рада (и тиме показати да сте рад пажљиво прочитали и разумели). Обим од 200 до 400 карактера.
U radu se kreće od uvoda u etiku koji obuhvata dosta podela i potpodela uz objasnjenje šta obuhvata koja. Predstavljaju
se glavni predstavnici i ideje koje su zastupali. Nakon toga priča se povezuje sa tehnologijom i računarima počevši od
korena njihovog odnosa. Dalje se predstavljaju pravila ponašanja na internetu. Na samom kraju imamo primere veze
između etike i računarstva u stvarnom svetu.

\section{Krupne primedbe i sugestije}
% Напишете своја запажања и конструктивне идеје шта у раду недостаје и шта би требало да се промени-измени-дода-одузме да би рад био квалитетнији.
Najpre bih istakao da sam sa lakoćom i zadvoljstvom pročitao zadati rad. Kritikovao bih zaključak koji i nije zaključak već 
kratka lista stavki koje su autori pre toga napisali. Zaključka nema pa seminarski rad ostaje bez završetka. Fali mu suština 
koja treba da ostane čitaocu iako sve ostalo zaboravi. Fali mu zaključak! \par

\odgovor{TODO. Obavezno prepraviti zaključak.}

Smatram da je dubina posvećena etici u računarstvu naspram dubine posvećene etičkoj teoriji mala i da je treba proširiti. 
Imajući u vidu da smo mi studenti Matematičkog Fakulteta koji svakodnevno koriste \textit{GNU/Linux} smer u kojem bih se ja kretao je tema 
slobodnog softvera. Čitajući listu dobih pravila ponašanja koju ste izneli u radu, a koju je napisala Asocijacija za kompjutersku 
mašineriju, stičem utisak da kompanije koje forsiraju vlasnički softver krše veći broj stavki. \par
Na kraju, vezano za materiju o kojoj ste pisali, mogli ste izneti kritiku za utilitarizam tj. 
bilo bi lepo da ste naglasiti čitaocu kompleksnost problema da se svakoj akciji 
nepristrasno i adekvatno dodeli težina koja bi kasnije uticala na sumu koja karakteriše problem.
Dalje u radu kažete da je u antičkoj Grčkoj najpoželjnija vrlina bila fizička hrabrost. Na šta se tu misli? Da li
je to Aristotelova vrlina ili Platonova? Navodite reference posle iznošenja tvrdnji. U sekciji gde obrađujete Kanta, tj. Kantijanizam, izneli ste neke činjenice o stvarima koje definiše i njihove defincije ali ne i reference.

\section{Sitne primedbe}
% Напишете своја запажања на тему штампарских-стилских-језичких грешки
U poslednjem pasusu sekcije 3.4 reč \textit{praktična} je loše napisana.
U drugom pasusu sekcije 4 zapeta nije adekvatno formatirana.

\odgovor{Obe greške su ispravljene.}

U drugom pasusu 5. sekcije druga rečenica treba da se prepravi jer ovako napisana nema smisla.

\odgovor{Nisam siguran na šta kolega misli. (TODO: Nek vidi neko: Te godine profesor Volter Maner (eng. \textit{Walter Maner}) je primetio da etički problemi na njegovom kursu medicinske etike budu teži ili se uveliko menjaju u slučajevima kada se računari pridodaju.)}

\section{Provera sadržajnosti i forme seminarskog rada}
% Oдговорите на следећа питања --- уз сваки одговор дати и образложење

\begin{enumerate}
\item Da li rad dobro odgovara na zadatu temu?\\
Zadati rad dobro dgovara na zadatu temu.
\item Da li je nešto važno propušteno?\\
Uvod i upoznavanje čitaoca sa etikom je temeljan. Međutim, konkretan deo posvećen računarskoj etici ostaje kratak i bez dovoljno dubine.
\item Da li ima suštinskih grešaka i propusta?\\
Ono što bih ja dodao i što smatram da fali su dve stvari. Prva, u deo u kome se prolazi kroz etiku bih dodao i Ničeove stavove jer ga smatram izuzetno važnim i uticajnim na polju filozofije morala. 
Premda je taj deo i ovako opsežan možda bi bilo previše. Druga stvar, deo oko etike računara je površno. Smernice u kojima bih se ja kretao navedene su u sekciji krupnih primedbi.
\item Da li je naslov rada dobro izabran?\\
Jeste.
\item Da li sažetak sadrži prave podatke o radu?\\
Sadrži.
\item Da li je rad lak-težak za čitanje?\\
Rad je izuzetno pitak. Pohvale autorima.
\item Da li je za razumevanje teksta potrebno predznanje i u kolikoj meri?\\
Ne, svaki pomenut pojam je adekvatno objašnjen.
\item Da li je u radu navedena odgovarajuća literatura?\\
Fali referenca na definiciju normativne etike kao i primenjene etike. U etici vrline iznošene su tvrdnje za 
najpoželjnije vrline bez ikakvih referenci.
\item Da li su u radu reference korektno navedene?\\
Imajući uvidu da niste koristili [8] u celosti navedite korišćene delove.
\item Da li je struktura rada adekvatna?\\
Jeste.
\item Da li rad sadrži sve elemente propisane uslovom seminarskog rada (slike, tabele, broj strana...)?\\
Fali naučni rad među referencama. Sve ostalo je u redu.

\odgovor{Naučni rad je referisan u poglavlju 5 ``Etika i računari''. U pitanju je rad ``Computer and Information Ethics'' - Terrel Bynum 2001. Nije očigledno na prvi pogled po priloženom url-u da je u pitanju naučni rad, ali u datom trenutku nismo uspeli da nađemo bolju veb adresu.(TODO: Napisati ovo kako treba + dodati url)}

\item Da li su slike i tabele funkcionalne i adekvatne?\\
Ispod slika nema izvor. Tabeli fali opis šta koja vrsta opisuje jer je druga vrsta neintuitivna.

\end{enumerate}

\section{Ocenite sebe}
% Napišite koliko ste upućeni u oblast koju recenzirate: 
% a) ekspert u datoj oblasti
% b) veoma upućeni u oblast
% c) srednje upućeni
% d) malo upućeni 
% e) skoro neupućeni
% f) potpuno neupućeni
% Obrazložite svoju odluku
Volim filozofiju i rado joj se posvećujem u slobodno vreme. Ipak filozofija je široka oblast stoga se ne smatram
kompetentnim na temu etike. Imajući ovo na umu rekao bih da sam \textbf{srednje upućen} u oblast koju recenziram.



\chapter{Dodatne izmene}
%Ovde navedite ukoliko ima izmena koje ste uradili a koje vam recenzenti nisu tražili. 

\end{document}
